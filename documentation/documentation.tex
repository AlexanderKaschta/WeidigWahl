\documentclass[ngerman]{ltxdoc}
\usepackage{babel}
\usepackage[T1]{fontenc}
\usepackage[utf8x]{inputenc}
%\OnlyDescription

\usepackage{parskip}
\usepackage{setspace}
\usepackage{xspace}
\onehalfspacing

\usepackage{etoolbox}
\usepackage{ifxetex}
\usepackage{ifluatex}

\ifboolexpr{bool {xetex} or bool {luatex}}{
  \usepackage{fontspec}
  \defaultfontfeatures{Ligatures=TeX}

  \newcounter{fontsnotfound}
  \newcommand{\checkfont}[1]{%
    \suppressfontnotfounderror=1%
    \font\x = "#1" at 10pt
    \selectfont
    \ifx\x\nullfont%
      \stepcounter{fontsnotfound}%
    \fi%
    \suppressfontnotfounderror=0%
  }

  \newcommand{\iffontsavailable}[3]{%
    \setcounter{fontsnotfound}{0}%
    \expandafter\forcsvlist\expandafter%
    \checkfont\expandafter{#1}%
    \ifnum\value{fontsnotfound}=0%
      #2%
    \else%
      #3%
    \fi%
  }
  \iffontsavailable{Fira Sans Light,%
                Fira Sans Light Italic,%
                Fira Sans,%
                Fira Sans Italic}{%
    \setmainfont[BoldFont={Fira Sans}]{Fira Sans Light}%
  }{%
    \iffontsavailable{Fira Sans Light OT,%
                  Fira Sans Light Italic OT,%
                  Fira Sans OT,%
                  Fira Sans Italic OT}{%
      \setmainfont[BoldFont={Fira Sans OT}]{Fira Sans Light OT}%
    }{%
      \typeout{%
        Could not find Fira Sans fonts. Creating documentation%
        with standard fonts.%
      }
    }
  }
  \iffontsavailable{Fira Mono, Fira Mono Bold}{%
    \setmonofont{Fira Mono}%
  }{%
    \iffontsavailable{Fira Mono OT, Fira Mono Bold OT}{%
      \setmonofont{Fira Mono OT}%
    }{%
      \typeout{%
        Could not find Fira Sans fonts. Creating documentation%
        with standard monospaced fonts.%
      }
    }
  }
}{
  \typeout{%
    You need to compile with XeLaTeX or LuaLaTeX to use the Fira fonts.%
  }
}

\usepackage{enumitem}
\setlist[itemize]{noitemsep}
\setlist[enumerate]{noitemsep}

\usepackage{xcolor}
\definecolor{mDarkBrown}{HTML}{604c38}
\definecolor{mDarkTeal}{HTML}{23373b}
\definecolor{mLightBrown}{HTML}{EB811B}
\definecolor{mLightGreen}{HTML}{14B03D}
\definecolor{mBackground}{HTML}{FFFFFF}

\usepackage{listings}

\lstdefinestyle{license}{
  basicstyle=\fontsize{7}{11}\selectfont\ttfamily,
  breaklines=true
}

\lstdefinestyle{tex}{
  language=[LaTeX]{TeX},
  basicstyle=\ttfamily,
  keywordstyle=\color{mLightBrown}\bfseries,
  commentstyle=\color{mLightGreen},
  stringstyle=\color{mLightGreen},
  backgroundcolor=\color{mBackground},
  numbers=none,
  numberstyle=\tiny\ttfamily,
  stepnumber=2,
  showspaces=false,
  showstringspaces=false,
  showtabs=false,
  frame=none,
  framerule=1pt,
  tabsize=2,
  rulesep=5em,
  captionpos=b,
  breaklines=true,
  breakatwhitespace=false,
  framexleftmargin=0em,
  framexrightmargin=0em,
  xleftmargin=0em,
  xrightmargin=0em,
  aboveskip=1em,
  belowskip=1em,
  morekeywords={usetheme,institute,maketitle,@metropolis@titleformat,%
                plain,setbeamercolor,metroset,setsansfont,setmonofont},
}


\lstMakeShortInline|
\usepackage{metalogo}

\usepackage[colorlinks=true,
            linkcolor=mLightBrown,
            menucolor=mLightBrown,
            pagecolor=mLightBrown,
            urlcolor=mLightBrown]{hyperref}

\newcommand{\DescribeOption}[4]{
  \DescribeMacro{#1}
  \begin{minipage}[t]{\textwidth}
    \textit{\textbf{\textcolor{mLightGreen}{#2}}}\dotfill\,#3\par
    \begingroup
    \vspace{0.5em}#4\par
    \endgroup
  \end{minipage}
}

\newcommand{\name}{\textbf{\textsc{WeidigWahl}}\xspace}

\title{Technische Dokumentation der Software \name}
\author{Alexander Kaschta \\ \url{https://github.com/AlexanderKaschta}}
\date{\today}

\begin{document}

\maketitle
\tableofcontents

\section{Einführung}

Dies ist die technische Dokumentation von \name. \name entstand im Ramen des
Informatikunterrichts und dient zur einfachen Durchführung von Kurswahlen an Schulen.
Die Anwendung ist webbasiert und läuft dann im Internet.

Die Schüler bzw. Teilnehmer kriegen am Anfang Zugangsdaten zu der Platform, wo
sie sich dann anmelden können. Dort finden sie allerlei Informationen zu der Wahl
und zu allen Kursen, die zur Auswahl stehen. Dort können die Schüler bis zu einer
festgelegten Deadline ihre Stimme oder Stimmen abgeben, welche Kurse sie gerne
besuchen würden. Sobald die Wahl dann abgelaufen ist, kann dann die Wahl
elektronisch ausgewertet werden.

Das System hat den Vorteil, dass man sich sehr viel Papier sparen kann. Denn es
wird keinerlei Papier mehr benötigt. Außerdem ist die Auswertung sehr einfach, da
man die Einteilung nicht selbst macht, sondern ein Algorithmus die Aufgabe übernimmt.
Dieser ist deutlich schneller und beachtet alle von den Schülern abgegebenen Wünsche.
Natürlich kann das System nicht alle Wünsche erfüllen, aber die Ursache der Problematik
liegt im Ursprung des Problems der Einteilung und würde auch bei einer manuellen
Wahl nicht verhindert werden können.

\section{Installation}

\subsection{Systemanforderungen}

Zur Installation von \name werden folgende Komponenten benötigt:

\begin{itemize}
  \item \href{http://httpd.apache.org/}{Apache HTTP Server} oder ähnlicher Server
  \item \href{https://www.php.net/}{PHP} 7+
  \item \href{https://mariadb.org/}{MariaDB} oder \href{https://www.mysql.com/}{MySQL}\footnote{Nicht getestet}
  \item PDO-Erweiterung für \href{https://www.php.net/}{PHP}
  \item PDO-MySQL-Treiber für \href{https://www.php.net/}{PHP}
  \item \href{https://www.phpmyadmin.net/}{phpMyAdmin}
  \item Ein Texteditor
\end{itemize}

Optional bzw. nach Bedarf:

\begin{itemize}
  \item \href{https://git-scm.com/}{Git}
  \item FTP-Client
\end{itemize}

\subsection{Installation}

Zum Beginn der Installation muss man sich eine Kopie des Software-Pakets besorgen.
Dies kann auf zwei Arten passieren. Man kann den Quellcode entweder mittels \href{https://git-scm.com/}{Git}
auf die eigene Festplatte klonen oder direkt als Zip-Paket vom
\href{https://github.com/AlexanderKaschta/WeidigWahl}{GitHub-Repository}\footnote{\url{https://github.com/AlexanderKaschta/WeidigWahl}} herunterladen.

Zum Download der Zip-Datei öffnet man den Link zum Repository im Internet-Browsers
seines Vertrauens. Dort muss man zuerst auf \glqq Clone or download \grqq und dann
auf \glqq Download Zip \grqq klicken. Nach dem Download muss die entsprechende
ZIP-Datei noch entpackt werden.

Für das Klonen mittels \href{https://git-scm.com/}{Git} muss nur folgender Befehl
in der Kommandozeile ausgeführt werden.

\begin{lstlisting}[style=tex]
$ git clone https://github.com/AlexanderKaschta/WeidigWahl.git
\end{lstlisting}

Nun öffnet man den Ordner mit dem Quellcode. Dort sollte man folgende Ordnerstruktur vorfinden:

\begin{multicols}{3}
  \begin{itemize}
    \item |core|
    \item |css|
    \item |docs|
    \item |documentation|
    \item |js|
    \item |sql|
    \item |uploads|
    \item |webfonts|
  \end{itemize}
\end{multicols}

\subsection{Datenbank}

Man benötigt eine Datenbank für \name. Dazu benötigen wir entweder \href{https://mariadb.org/}{MariaDB}
oder \href{https://www.mysql.com/}{MySQL}. Das ganze wurde mittels MariaDB entwickelt,
da diese kompatibel zu MySQL sollte auch diese funktionieren. Dies wurde bis jetzt
jedoch noch nicht getestet. Man könnte auch noch eine andere Datenbank verwenden,
jedoch bräuchte man dafür den entsprechenden PDO-Treiber und müsste alle SQL-Befehle
anpassen, sodass diese kompatibel mit der entsprechende Datenbank sind.

Es kann für \name eine vorhandene Datenbank benutzt werden, wenn folgende Tabellen
noch erstellt werden können bzw. noch nicht benutzt werden. Natürlich kann man
auch eine komplett neue Datenbank erstellen.

\begin{multicols}{3}
  \begin{itemize}
    \item |tbl_users|
    \item |tbl_sportwahl|
    \item |tbl_kurse|
    \item |tbl_teilnehmer|
    \item |tbl_ergebnisse|
  \end{itemize}
\end{multicols}

Die entsprechende Datenbank wird dann mit \href{https://www.phpmyadmin.net/}{phpMyAdmin} geöffnet. Nun müssen die benötigten
Tabellen erstellt werden. Alle SQL-Befehle zum erstellen der entsprechende Tabellen sind
in der Datei |sql/setup.sql| dokumentiert. Diese Befehle kopiert man dann und führt sie
über \href{https://www.phpmyadmin.net/}{phpMyAdmin} aus. Am Ende sollte die Datenbank
dann alle fünf Tabellen enthalten.

Nun muss man jetzt nur noch einen Benutzer mit Administratorrechten erstellen. Dazu muss
dieser SQL-Befehl ausgeführt werden. Benutzername als auch das Passwort sollten
dazu geändert werden.

\begin{lstlisting}[language=SQL, breaklines=true]
INSERT INTO tbl_users (vorname, nachname, benutzername, geburtsdatum, passwort, ist_aktiv, jahrgang, klasse, datum_erstellt, datum_letzte_anderung, kann_reset_anfordern, ist_admin) VALUES
('Max', 'Mustermann', 'admin', '1970-01-01', 'passwort', 1, 99, 'Admins', NOW(), NOW(), 1, 1);
\end{lstlisting}

\subsection{Konfiguration}

Für die gesamte Konfiguration von \name gibt es eine zentrale Konfigurationsdatei.
Diese heißt |config.php| und liegt im |core/config.php|. In dieser Datei sind alle
für den Betrieb notwendige Variablen abgebildet und werden dort entsprechend entnommen.

\DescribeOption{Option}{Name der Option}{Standardwert}{
  Eine kurze Beschreibung der Option.
}

Dabei ist die Eigenschaft im PHP-Code wie folgt abgebildet:

\begin{lstlisting}[language=PHP, breaklines=true]
define("Option", "Standardwert");
\end{lstlisting}

Folgende Eigenschaften gibt es:

\DescribeOption{DB\_HOST}{Hostadresse der Datenbank}{localhost}{
  In diesem Feld wird die Adresse zum Hostserver der Datenbank eingetragen. Da
  sehr häufig der Webserver auf einem anderem Server läuft als die Datenbank, so
  muss man passend angeben, wo die Datenbank zu erreichen ist. Der Standardwert
  |localhost| ist die Hostadresse zum eigenem Rechner. Sie ist äquivalent zu
  |127.0.0.1| .
}

\DescribeOption{DB\_USER}{Benutzername für die Datenbank}{root}{
  Der Zugang zu einer Datenbank benötigt Zugangsdaten. Dabei handelt es sich um
  einen Benutzernamen und das dazu passende Passwort. Bei \href{https://mariadb.org/}{MariaDB}
  oder \href{https://www.mysql.com/}{MySQL} ist der Standartbenutzername nach der
  Installation immer |root|, falls dies nicht geändert worden ist oder noch ein
  weiterer Benutzer zur Datenbank hinzugefügt worden ist.
}

\DescribeOption{DB\_PASSWORT}{Passwort für die Datenbank}{\glqq\grqq}{
  Der Zugang zu einer Datenbank benötigt Zugangsdaten. Dabei handelt es sich um
  einen Benutzernamen und das dazu passende Passwort. Bei \href{https://mariadb.org/}{MariaDB}
  oder \href{https://www.mysql.com/}{MySQL} ist das Standartpasswort für den Benutzer
  |root| nach der Installation immer ein leere Zeichenkette. Falls dieses Passwort
  geändert worden ist oder ein anderer Benutzer benutzt wird, dann muss dieses entsprechend
  angepasst werden.
}

\DescribeOption{DB\_NAME}{Name für die Datenbank}{weidigwahl}{
  Dies ist der Name der Datenbank, welche die vorher erstellt Tabellen enthält.
  Da ein Datenbankserver mehrere Datenbank bereitgestellen kann, muss immer angegeben
  werden, um welche Datenbank es sich konkret handelt.
}

\DescribeOption{PROJECT\_NAME}{Anzeigename für Titelleiste}{\name}{
Diese Option spezifiziert den Namen der den Nutzer in der Titelleiste des Browsers
bei der Benutzung angezeigt wird. Diese Eigenschaft muss nicht zwingen geändert
werden.
}

\DescribeOption{PROJECT\_VERSION}{Versionsnummer der Software}{1.0.0}{
Dies ist die Versionsnummer der aktuell installierten Software. Diese wird nur
intern verwendet und \textbf{sollte nicht} geändert werden.
}

\DescribeOption{ADMINISTRATOR}{Name des Administrators}{Hr. Mustermann}{
Dies ist der Name des Administrators, welcher auf der Hauptseite angezeigt wird.
Er ist der Ansprechpartner an dem man sich wenden kann, falls man Probleme hat
oder irgendwelche unerwarteten Fehler aufgetreten sind.
}

Sobald die Konfigurationsdatei entsprechend den Bedürfnissen mit einem Texteditor
angepasst worden ist, kann man die Software bereitgestellt\footnote{veröffentlichen}.

\subsection{Veröffentlichung}

Für die Veröffentlichung muss nun einfach die gesamte Ordnerstruktur, die am Anfang
herunterladen geladen worden ist, einfach nur noch auf den Server geladen werden.
Wenn es ein Server ist, der auf dem eigenem System läuft, dann reicht es aus, den
kompletten Inhalt des Ordners in den Serverordner zu kopieren. Falls man jedoch einen
externen Server besitzt, dann wird man die Dateien höchstwahrscheinlich mittels
FTP transferieren.

\textbf{Wichtige Anmerkung}: Die Ordner |sql| und |documentation| werden für den produktiven Betrieb nicht
benötigt. Damit man ein sicheres System hat wird empfohlen die beiden Ordner nicht
auf den Server zu laden oder wenn sie schon auf dem Server sind, sie von dort zu löschen.
Dies ist Teil von \href{https://de.wikipedia.org/wiki/Security_through_obscurity}{Security by obscurity}.
Falls bei der Bereitstellung auch noch Ordner mit den Namen |.idea| oder |.git| auftauchen,
dann können Sie diese auch entfernen. Diese werden nur für die Weiterentwicklung
der Software benutzt und sind für die produktive Nutzung überflüssig.

Nachdem man das gesamte System veröffentlicht hat, sollte man nochmal kontrollieren,
ob man sich auf der Seite anmelden kann und alles funktioniert.

\section{Nutzung}

In diesem Kapitel der Dokumentation werden nun die einzelnen Aspekte und Ansichten
in der Software beschreiben und dokumentiert. Dies sollte einem einen entsprechend
guten Überblick über die Software geben.

Alle Parameter werden in folgendem Format dargestellt:

\DescribeOption{Stichwort}{Name}{Beispielwert für die Sportwahl}{
  Zusammenfassung des Parameters inklusive seiner Bedeutung.
}

\subsection{Anmeldeseite}

Wenn man \name aufruft, so kommt man am Anfang immer zur Anmeldeseite, wo man sich
anmelden kann. Auf der Anmeldeseite kann man sich sowohl als normaler Nutzer als
auch Administrator anmelden. Denn nach der entsprechend erfolgreichen Anmeldung
wird man auf die entsprechend passende Seite automatisch weitergeleitet.

Auf der Seite gibt es ein Feld für den Benutzernamen und ein weiteres für das
Passwort des Nutzers. Darunter gibt es dann den Button\footnote{deut. Knopf}, womit
man die Anmeldung ausführen kann.

Falls man sein Passwort vergessen hat, dann kann man unten auf einen Hyperlink
mit der Aufschrift \glqq Accountdaten vergessen\grqq drücken. Dies lässt eine kleine
Box unten erscheinen, wo ein Hinweis dazu steht. Dazu soll man entweder seinen
zuständigen Lehrer fragen oder sich an den Administrator der Platform wenden.

Die Anmeldeseite kann auch Fehler ausgeben. Diese erscheinen immer unter der
Navigationsleiste mit \name. Diese sind \textcolor{red}{\textbf{rot}} hinterlegt
und enthalten entsprechend Textmeldungen. Falls ein Fehler bei der Anmeldung auftritt,
dann bekommt man eine Meldung, dass die eingegebenen Daten nicht gefunden werden konnten.
Andere Fehler sorgen für andere Fehlermeldungen, jedoch sind diese so verfasst,
dass diese auf der Anmeldeseite relativ wenige Informationen ausgeben. Falls ein
interner Fehler auftritt, dann kann es mehrere Ursachen haben. Entweder funktioniert
die Verbindung zur Datenbank nicht oder andere Parameter wurde nicht übertragen.

Es kann durchaus dazu kommen, dass wenn man eine Seite der \name direkt aufruft
ohne die entsprechend Vorgaben zu erfüllen, dass dann die Session von einem
gelöscht wird und man auf die Anmeldeseite automatisch weitergeleitet wird.
Genaueres dazu ist im Abschnitt |Seitenpolicy| zu finden.

\subsubsection{Abmeldung}

Sobald man sich angemeldet hat, sieht man auf jeder der Seiten die Navigationsleiste
mit der Aufschrift \name. Ganz rechts auf der Navigationsleiste steht der Text
\glqq Logout\grqq, womit man sich abmelden kann.

\subsection{Hauptseite}

Die Hauptseite bekommt man nach der Anmeldung zu gesicht. Für einen Administrator
sind dort alle Kurswahlen aufgelistet, die es gibt. Dazu gibt es noch zwei spezielle
Einträge mit \glqq Neue Wahl!\grqq und \glqq Schüler\grqq. Bei der ersten Kachel
kann man eine neue Kurswahl erstellen. Dies ist im Abschnitt |Wahl erstellen|
dokumentiert. Mit der zweiten Kachel gelangt man zur |Teilnehmerverwaltung|.
Dort kann man alle Nutzer der Platform einsehen und verwalten. Nur Nutzer mit
Administratorrechten können auf die Teilnehmerverwaltung zugreifen und Wahlen erstellen,
bearbeiten, auswerten und löschen.

Normale Nutzer bekommen nur eine Liste der Kurswahlen, an denen sie teilnehmen oder
schon teilgenommen haben. Falls man an keiner Kurswahl teilnimmt dann bekommt
man den Text angezeigt, dass für einen keine Wahlen zur Verfügung stehen.

\subsection{Wahl erstellen}

Eine Kurswahl benötigt folgende Parameter:

\DescribeOption{Name}{Name der Kurswahl}{Sportwahl 2020}{
  Dies ist der Name der Kurswahl. Dieser wird dann auf der Kachel der Hauptseite
  als auch an allen anderen Stellen angezeigt. Er dient dazu diese Wahl von
  vergangenen Wahlen oder anderen Wahlen zu unterscheiden.
}

\DescribeOption{Beschreibung}{Beschreibung der Kurswahl}{Sportkurse für E1 bis Q4}{
  Dies ist die Beschreibung der Kurswahl. Diese sollte kurz die Kurswahl beschreiben und
  den Namen der Kurswahl unterstützen.
}

\DescribeOption{Wunschabgaben}{Anzahl der Schülerwünsche}{3}{
  Die Anzahl der Schülerwünsche ist eine natürliche Zahl größer |0| und gibt an,
  wie viele Wünsche jeder Schüler abgeben kann. Diese Zahl darf nicht kleiner
  als die |Kurszuteilung| sein. Empfohlen wird außerdem, dass die Anzahl der Schülerwünsche
  kleiner als die Anzahl der Kurse für die Wahl ist, da dies an anderen Stellen im
  System Probleme verursacht.
}

\DescribeOption{Kurszuteilung}{Anzahl der Kurse, die ein Teilnehmer bekommen soll}{1}{
  Die Anzahl der Schülerwünsche ist eine natürliche Zahl größer |0| und gibt an,
  wie viele Kurse ein Schüler eingeteilt werden soll. Für die meisten Kurswahlen
  reicht hier |1|, da jedoch auch Wahlen mit mehreren Kursen pro Nutzer möglich sein
  sollen, muss dies hier angegeben werden.
}

\DescribeOption{Wahlbeginn}{Beginn der Wahl}{10.08.2020}{
  Dies ist der Zeitpunkt ab dem eine Wahl für die Nutzer zugänglich ist. Der Nutzer
  bekommt dennoch erst Zugriff zu der Wahl, wenn diese auch aktiviert ist, sodass
  trotz des angegebenen Datums die Teilnehmer erst später auf die Wahl zugreifen
  können. Die Teilnehmer bekommen eine Wahl zu sehen, sobald diese aktiviert ist
  und der Teilnehmer der entsprechenden Kurswahl zugeteilt ist. Dennoch können
  die Nutzer erst ab dem Beginn der Wahl ihre Stimmen abgeben. Das Datum des |Wahlbeginns|
  muss kleiner als das Datum des |Wahlendes| sein.
  Empfehlung: Geben sie jeder Wahl ca. eine Woche Zeit zur Stimmabgabe. Jedoch
  kann diese schon früher erstellt und aktiviert werden, sodass die Schüler auch
  schon früher Einsicht in die Wahl bekommen können.
}

\DescribeOption{Wahlende}{Ende der Wahl}{19.08.2020}{
  Das Ende der Wahl ist der Zeitpunkt, an dem die Wahl beendet. Dies passiert
  um 23:59, sodass die Teilnehmer den ganzen Tag noch nutzen können, um ihre Stimme
  abzugeben oder ihren Stimmwunsch zu ändern. Der Zeitpunkt für das |Wahlende| muss
  größer als der Zeitpunkt für den |Wahlbeginn| sein.
}

\subsection{Teilnehmerverwaltung}

In der Teilnehmerverwaltung sieht man eine Liste mit allen Nutzern, die in dem
System aktuell eingetragen sind. Dies sind nicht nur die Teilnehmer sondern auch
alle Administratoren. In dieser Liste sind folgende Informationen aufgelistet:

\begin{multicols}{3}
  \begin{itemize}
    \item |Nutzer-ID|
    \item |Vorname|
    \item |Nachname|
    \item |Benutzername|
    \item |Passwort|\footnote{Hier im Klartext}
    \item |Aktiviert?|\footnote{Zeigt an, ob der Account aktiviert ist}
    \item |Admin?|\footnote{Besitzt der Nutzer Administratorrechte}
    \item |Jahrgang|
    \item |Klasse|
  \end{itemize}
\end{multicols}

Hinter jedem Nutzereintrag gibt es noch zwei Aktionsbutton. Der erste dient zur
Bearbeitung des Nutzers und mit dem zweiten Button kann man den Nutzer entsprechend
löschen. Diesen Nutzer kann man nur dann löschen, wenn dieser Nutzer nicht der
eigene Nutzer ist. Dies soll verhindern, dass man als Administrator sich seinen
Zugang zur Platform nicht versperrt.

Zusätzlich gibt es am unteren Ende der Liste die Möglichkeiten neue Nutzer zu erstellen oder Nutzer
automatisch aus einer CSV-Datei zu importieren. Ganz oben auf der Seite gibt es
jedoch einen \textcolor{red}{\textbf{roten Button}} mit dem man alle Teilnehmer
löschen kann, die keine Administratorrechte besitzen. Somit kann man mit dieser
Funktion die Nutzerliste vor einem neuen Schuljahr aufräumen.

\textit{Technische Notiz}: Es werden hier nur die Einträge auf der Tabelle |tbl_users|
gelöscht, sodass im Fehlerfall man immer noch die Wahlergebnisse als auch die
Teilnahmen des Nutzers besitzt.

\subsection{Nutzer erstellen}

Dieses Formular dient zur Erstellung von neuen Nutzern als auch der Bearbeitung
von schon vorhandenen Nutzern. Dabei läuft beides bis auf eine Ausnahme identisch
ab. Ein Nutzer wird nur dann erstellt oder aktualisiert, wenn die angegebenen Daten
auch passen. Wenn irgendwo ein Fehler vorliegt, dann wird dieser angezeigt und man
kann die Daten entsprechend korrigieren. Bei der Bearbeitung eines Nutzers werden
die Änderungen nur dann übernommen, wenn man dies auch bestätigt. Man kann die Bearbeitung
also jederzeit ohne jegliche Auswirkungen auf das System abbrechen.

Folgende Informationen sind wichtig für die Erstellung und Bearbeitung von Nutzern:

\DescribeOption{Vorname}{Vorname des Nutzers}{Max}{
  Dies ist der Vorname eines Nutzers. Dieser darf nicht leer sein.
}

\DescribeOption{Nachname}{Nachname des Nutzers}{Mustermann}{
  Dies ist der Nachnamen eines Nutzers. Dieser darf nicht leer sein.
}

\DescribeOption{Benutzername}{Benutzername}{Max.Mustermann}{
  Dies ist der Benutzername, mit dem sich der Nutzer dann anmelden wird.
}

\DescribeOption{Geburtsdatum}{Geburtsdatum des Nutzers}{01.01.2020}{
  Dies ist das Geburtsdatum des Nutzers. Falls es zwei Nutzer mit dem gleichen
  Namen gibt, dann sollte dies in dem unwahrscheinlichen Fall das entsprechende
  Unterscheidungskriterium sein.
}

\DescribeOption{Passwort}{Passwort für den Nutzer}{passwort}{
  Der Nutzer benötigt für die Anmeldung ein Passwort. Das Passwort darf nicht
  leer sein und sollte mindestens |8| Zeichen lang sein.

  \textbf{Anmerkung}: Das hier gezeigt Beispielpasswort ist in keinerlei Hinsicht
  sicher und sollte niemals verwendet werden.
}

\DescribeOption{Aktiv}{Ist der Account aktiv?}{Ja}{
  Dies ist eine Checkbox, womit mit man einstellen
  kann, ob der Account wirklich aktiv ist. Wenn ein Haken da ist, dann ist bzw.
  soll der Account aktiv sein. Wenn dort kein Haken zu sehen ist, dann ist der
  Account nicht aktiv.
}

\DescribeOption{Jahrgang}{Jahrgang des Nutzer}{10}{
  Für die Nutzer gibt es eine Möglichkeit den Jahrgang anzugeben. Diese Feld muss
  auch entsprechend befüllt werden. Es dient dazu, dass man ganze Gruppen von
  Schülern einfach zu einer Kurswahl hinzufügen kann. Dabei muss jedoch dafür
  gesorgt werden, dass alle Jahrgangseinträge für den gleichen Jahrgang identisch
  sind, denn nur dann funktioniert es einwandfrei. Das auch Administratoren
  einen Jahrgang benötigen ist hier der \textit{Tipp}: Einfach eine entsprechend
  große Nummer, wie die |100| oder so, nehmen und diese allen Administratoren
  geben. Diese können entsprechend identifiziert werden und es fällt auf, dass
  dies kein gewöhnlicher Jahrgang ist.
}

\DescribeOption{Klasse}{Klasse des Nutzer}{10E}{
  Für die Nutzer gibt es eine Möglichkeit die Klasse anzugeben. Diese Feld muss
  auch entsprechend befüllt werden. Es dient dazu, dass man ganze Gruppen von
  Schülern einfach zu einer Kurswahl hinzufügen kann. Dabei muss jedoch dafür
  gesorgt werden, dass alle Klassenseinträge für die gleiche Klasse identisch
  sind, denn nur dann funktioniert es einwandfrei. Das auch Administratoren
  eine Klasse benötigen ist hier der \textit{Tipp}: Hier einfach eine Text wie
  |Admins| verwenden und schon hat man einen passenden Klassennamen.
}

\DescribeOption{Administrator}{Soll dieser Nutzer Administratorrechte besitzen}{Nein}{
  Dies ist eine Checkbox, womit mit man einstellen kann, ob der Account
  Administratorrechte besitzen soll. Wenn ein Haken da ist, dann ist bzw.
  soll der Account diese Rechte haben. Wenn dort kein Haken zu sehen ist, dann ist der
  Account ein normaler Nutzeraccount.
}

\subsubsection{CSV-Import}

Der CSV\footnote{Comma separated values - Durch Kommas getrennte Werte}-Import ist dazu entwickelt worden, dass man sehr schnell eine große
Anzahl von Schülern in das System eintragen kann. Dies kann dadurch sinnvoll sein,
dass es bereits entsprechende Listen mit den Daten gibt und diese nun damit
genutzt werden können.

Damit man Schüler importieren kann, benötigt man eine |CSV-Datei| mit folgender
Formattierung:

\begin{lstlisting}[style=tex]
<Vorname>;<Nachname>;<Geburtsdatum>;<Jahrgang>;<Klasse/TG>
\end{lstlisting}

Wenn man sich für den CSV-Import entschieden hat, dann wird zu einem Upload-Formular
weitergeleitet. Dort muss man die CSV-Datei entsprechend auswählen und hochladen.
Es werden zwar in der Auswahl alle Dateitypen erlaubt, jedoch nimmt hier das
System nur |txt|- und |csv|-Dateien an. Andere Dateitypen werden für entsprechende
Fehlermeldungen sorgen.

Nachdem Upload wird die Datei im Ordner |uploads| zwischengespeichert und verarbeitet.
Dazu wird die Datei durchgeganen und jeder Nutzer in die Datenbank eingetragen.
Der Benutzername als auch das Passwort werden automatisch generiert. Diese
Accounts sind standardmäßig aktiviert und besitzen \textbf{keine} Administratorrechte.

Sobald die Verarbeitung erfolgreich durchgeführt worden ist, wird die ursprüngliche
hochgeladene und im Ordner |uploads| gespeicherte |CSV-Datei| gelöscht, sodass
keinerlei Rückstände zurückbleiben. 


\subsection{Seitenpolicy}

Das System gibt Meldungen aus. Davon gibt es zwei Arten. Fehlermeldungen werden
angezeigt, wenn ein Fehler aufgetreten ist. Diese sind \textcolor{red}{\textbf{rot}} hinterlegt.
Zusätzlich gibt es noch Erfolgsmeldung, die angezeigt werden, wenn eine Operation
erfolgreich durchgeführt worden ist. Diese sind \textcolor{green}{\textbf{grün}} hinterlegt.

Falls man eine Seiten der \name ohne gültige Anmeldung aufruft, dann wird man immer
auf die Anmeldeseite weitergeleitet. Das gleiche passiert auch, wenn auf einer der
Seiten ein Fehler auftritt.

Es können auch Nutzerfehler auftreten. Das kann ein fehlende oder eine ungültige
Eingabe sein. Diese Fehler behandelt das System und gibt entsprechend Rückmeldung
an den Nutzer. Dazu wird er auf die ursprüngliche Seite zurückgeleitet und sieht dort
die Meldung.

\section{Mitwirkende}

Dieses Projekt wurde in Zusammenarbeit mit mehreren Personen entwickelt, jedoch werden diese aus Datenschutzgrüden
her nicht veröffentlicht. Trotzdem ein Dankeschön an alle, die an diesem Projekt
beteiligt waren.

\section{Lizenzen}

\name ist unter der MIT Lizenz veröffentlicht:

\lstinputlisting[style=license]{LICENSE.txt}

Da diese Software Komponenten enthält und benutzt, welche anderer Personen oder
Entwickler erstellt haben, müssen auch deren Lizenzen aufgelistet werden. Diese
sind nun in den Unterabschnitten aufgelistet. Alle Angaben sind ohne Gewähr.

%Auflistung aller Teilkomponenten
\subsection{Bootstrap 4.1}

\lstinputlisting[style=license]{LICENSE_Bootstrap.txt}

\subsection{Font Awesome}

Die Lizenz ist in mehrere Teile unterteilt, die hier in den entsprechenden Unterpunkten
gelistet werden. Die komplette Lizenz ist hier zu finden:

\url{https://fontawesome.com/license/free}.

\subsubsection*{Icons}
CC BY 4.0 License (\url{https://creativecommons.org/licenses/by/4.0/})

\begin{lstlisting}[style=license]
In the Font Awesome Free download, the CC BY 4.0 license applies to all icons
packaged as SVG and JS file types.
\end{lstlisting}

\subsubsection*{Fonts}
SIL OFL 1.1 License (\url{https://scripts.sil.org/OFL})

\begin{lstlisting}[style=license]
In the Font Awesome Free download, the SIL OFL license applies to all icons
packaged as web and desktop font files.
\end{lstlisting}

\subsubsection*{Code}
MIT License (\url{https://opensource.org/licenses/MIT})

\begin{lstlisting}[style=license]
In the Font Awesome Free download, the MIT license applies to all non-font and
non-icon files.
\end{lstlisting}

\subsection{tFPDF}

Original author: Ian Back

Maintainer: Tycho Veltmeijer

License: LGPLv3

\lstinputlisting[style=license]{lgpl-3.0.txt}

\subsection{Popper.js}

\lstinputlisting[style=license]{LICENSE_Popper.txt}

\subsection{jQuery}

\lstinputlisting[style=license]{LICENSE_jQuery.txt}

\subsection{Chart.js}

\lstinputlisting[style=license]{LICENSE_ChartsJs.txt}

\section{Vorlage}

Die Dokumentation verwendet die Schriftart
\href{https://www.mozilla.org/en-US/styleguide/products/firefox-os/typeface/}
{Fira Sans}, welche von Mozilla bereitgestellt wird und von
\href{http://www.carrois.com/fira-3-1/}{Carrois} erstellt worden ist.

Als Vorlage für diese Dokumentation diente die Dokumentation des Projekts \textbf{mtheme},
welches auf \href{https://github.com/matze/mtheme}{GitHub} gefunden werden kann.

\end{document}
